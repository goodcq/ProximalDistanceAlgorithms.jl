\documentclass{article}

\usepackage{amsbsy,amsthm,amsmath,amssymb}
%\usepackage{graphicx,natbib,booktabs}
\usepackage{graphicx}
\usepackage[margin=1in]{geometry}
\usepackage{subfig}
\usepackage{algorithm}
\usepackage{algpseudocode}
\usepackage{paralist}

\newtheorem{proposition}{Proposition}
\newtheorem{lemma}{Lemma}
\newtheorem{example}{Example}
\newtheorem{definition}{Definition}
\def\E{\mathop{\rm E\,\!}\nolimits}
\def\Var{\mathop{\rm Var}\nolimits}
\def\Cov{\mathop{\rm Cov}\nolimits}
\def\den{\mathop{\rm den}\nolimits}
\def\midd{\mathop{\,|\,}\nolimits}
\def\sgn{\mathop{\rm sgn}\nolimits}
\def\vec{\mathop{\rm vec}\nolimits}
\def\sinc{\mathop{\rm sinc}\nolimits}
\def\curl{\mathop{\rm curl}\nolimits}
\def\div{\mathop{\rm div}\nolimits}
\def\tr{\mathop{\rm tr}\nolimits}
\def\len{\mathop{\rm len}\nolimits}
\def\dist{\mathop{\rm dist}\nolimits}
\def\prox{\mathop{\rm prox}\nolimits}
\def\amp{\mathop{\:\:\,}\nolimits}
\def\Real{\mathop{\mathbb{R}}\nolimits}
\def\dom{\mathop{\bf dom}\nolimits}
\def\argmin{\mathop{\rm argmin}\nolimits}
\def\argmax{\mathop{\rm argmax}\nolimits}
\newcommand{\svskip}{\vspace{1.75mm}}
\newcommand{\ba}{\boldsymbol{a}}
\newcommand{\bb}{\boldsymbol{b}}
\newcommand{\bc}{\boldsymbol{c}}
\newcommand{\bd}{\boldsymbol{d}}
\newcommand{\be}{\boldsymbol{e}}
\newcommand{\bff}{\boldsymbol{f}}
\newcommand{\bg}{\boldsymbol{g}}
\newcommand{\bh}{\boldsymbol{h}}
\newcommand{\bi}{\boldsymbol{i}}
\newcommand{\bj}{\boldsymbol{j}}
\newcommand{\bk}{\boldsymbol{k}}
\newcommand{\bl}{\boldsymbol{l}}
\newcommand{\bm}{\boldsymbol{m}}
\newcommand{\bn}{\boldsymbol{n}}
\newcommand{\bo}{\boldsymbol{o}}
\newcommand{\bp}{\boldsymbol{p}}
\newcommand{\bq}{\boldsymbol{q}}
\newcommand{\br}{\boldsymbol{r}}
\newcommand{\bs}{\boldsymbol{s}}
\newcommand{\bt}{\boldsymbol{t}}
\newcommand{\bu}{\boldsymbol{u}}
\newcommand{\bv}{\boldsymbol{v}}
\newcommand{\bw}{\boldsymbol{w}}
\newcommand{\bx}{\boldsymbol{x}}
\newcommand{\by}{\boldsymbol{y}}
\newcommand{\bz}{\boldsymbol{z}}
\newcommand{\bA}{\boldsymbol{A}}
\newcommand{\bB}{\boldsymbol{B}}
\newcommand{\bC}{\boldsymbol{C}}
\newcommand{\bD}{\boldsymbol{D}}
\newcommand{\bE}{\boldsymbol{E}}
\newcommand{\bF}{\boldsymbol{F}}
\newcommand{\bG}{\boldsymbol{G}}
\newcommand{\bH}{\boldsymbol{H}}
\newcommand{\bI}{\boldsymbol{I}}
\newcommand{\bJ}{\boldsymbol{J}}
\newcommand{\bK}{\boldsymbol{K}}
\newcommand{\bL}{\boldsymbol{L}}
\newcommand{\bM}{\boldsymbol{M}}
\newcommand{\bN}{\boldsymbol{N}}
\newcommand{\bO}{\boldsymbol{O}}
\newcommand{\bP}{\boldsymbol{P}}
\newcommand{\bQ}{\boldsymbol{Q}}
\newcommand{\bR}{\boldsymbol{R}}
\newcommand{\bS}{\boldsymbol{S}}
\newcommand{\bT}{\boldsymbol{T}}
\newcommand{\bU}{\boldsymbol{U}}
\newcommand{\bV}{\boldsymbol{V}}
\newcommand{\bW}{\boldsymbol{W}}
\newcommand{\bX}{\boldsymbol{X}}
\newcommand{\bY}{\boldsymbol{Y}}
\newcommand{\bZ}{\boldsymbol{Z}}
\newcommand{\balpha}{\boldsymbol{\alpha}}
\newcommand{\bbeta}{\boldsymbol{\beta}}
\newcommand{\bgamma}{\boldsymbol{\gamma}}
\newcommand{\bdelta}{\boldsymbol{\delta}}
\newcommand{\bepsilon}{\boldsymbol{\epsilon}}
\newcommand{\blambda}{\boldsymbol{\lambda}}
\newcommand{\bmu}{\boldsymbol{\mu}}
\newcommand{\bnu}{\boldsymbol{\nu}}
\newcommand{\bphi}{\boldsymbol{\phi}}
\newcommand{\bpi}{\boldsymbol{\pi}}
\newcommand{\bsigma}{\boldsymbol{\sigma}}
\newcommand{\btheta}{\boldsymbol{\theta}}
\newcommand{\bomega}{\boldsymbol{\omega}}
\newcommand{\bxi}{\boldsymbol{\xi}}
\newcommand{\bGamma}{\boldsymbol{\rho}}
\newcommand{\bDelta}{\boldsymbol{\Delta}}
\newcommand{\bTheta}{\boldsymbol{\Theta}}
\newcommand{\bLambda}{\boldsymbol{\Lambda}}
\newcommand{\bXi}{\boldsymbol{\Xi}}
\newcommand{\bPi}{\boldsymbol{\Pi}}
\newcommand{\bOmega}{\boldsymbol{\Omega}}
\newcommand{\bUpsilon}{\boldsymbol{\Upsilon}}
\newcommand{\bPhi}{\boldsymbol{\Phi}}
\newcommand{\bPsi}{\boldsymbol{\Psi}}
\newcommand{\bSigma}{\boldsymbol{\Sigma}}
%% Matrix-matrix operations
\newcommand{\Kron}{\otimes} %Kronecker
%\newcommand{\Khat}{\odot} %Khatri-Rao
\newcommand{\Hada}{\ast} %Hadamard
%\newcommand{\Divide}{\varoslash}
%% Norms
\newcommand{\abs}[1]{\lvert{#1}\rvert}

\usepackage{hyperref}

\title{Proximal Distance Application: Convex Clustering}
\author{Alfonso Landeros}
\date{\today}

\begin{document}
\maketitle

Convex clustering of \(n\) samples based on \(d\) features can be formulated in terms of the regularized objective
\begin{equation}
    \label{eq:regularized-objective}
    F_{\gamma}(\bU)
    =
    \frac{1}{2} \|\bU - \bX\|_{F}^{2}
    +
    \gamma \sum_{i < j} w_{ij} \|\bU (\be_{i} - \be_{j})\|,
\end{equation}
where \(\bX \in \Real^{d \times n}\) encodes the data, columns of \(\bU \in \Real^{d \times n}\) represent cluster assignments, and \(\be_{k} \in \Real^{n}\) is the standard basis vector.
The weights \(w_{ij}\) have a graphical interpretation.
Related samples have \(w_{ij} > 0\), otherwise \(w_{ij} = 0\).
This implies that minimization of \(F_{\gamma}(\bU)\) separates over the connected components of the graph.
Finally, the regularization parameter \(\gamma\) tunes the number of clusters in a non-linear fashion.
Previous work establishes that the solution path \(\bU(\gamma)\) varies continuously with respect to regularization \cite{chi2015}.

The convex clustering problem can be solved with an ADMM approach, or its relative the AMA.
These notes seek to reformulate convex clustering as a projection problem.
Enforcing a small number of clusters amounts to strong consensus in columns of \(\bU\).
An equivalent approach is to impose \textit{block} sparsity on a vector (or matrix) encoding the differences \(\bu_{i} - \bu_{j}\).
The advantages of the projection perspective include access to proximal distance methods and an explicit mechanism for selecting the number of clusters.

\section*{\center The Fusion Matrix}

Here we derive a linear operator \(\bD\) mapping columns of \(\bU\) to the differences \(\bu_{i} - \bu_{j}\).
This operator will be referred to as the \textit{fusion matrix} for the convex clustering problem.

To start, define a collection of comparison matrices \(\bD^{i,j} \in \Real^{d \times dn}\) by the rule
\begin{equation}
    \label{eq:comparison-matrix-1}
    \bD^{i,j}[\vec(\bU)] = \bu_{i} - \bu_{j},
\end{equation}
with \(\vec(\bU) \in \Real^{dn}\).
Note that the vectorization operation maps column \(\bu_{j}\) to the linear indices \((j-1)n + 1\) up to \((j-1)n + d\).
Thus, define the index sets
\begin{equation*}
    \label{eq:index-sets}
    \mathcal{I}_{j}^{n}
    =
    \{k : (j-1)n + 1 \le k \le (j-1)n + d\}
\end{equation*}
so that rows of \(\bD^{i,j}\) take the form
\begin{equation}
    \label{eq:comparison-matrix-2}
    D_{k\ell}^{i,j}
    =
    \begin{cases}
        +1, & \text{if \(\ell \in \mathcal{I}_{i}^{n}\)} \\
        -1, & \text{if \(\ell \in \mathcal{I}_{j}^{n}\)} \\
        \phantom{+}0, & \text{otherwise}.
    \end{cases}
\end{equation}
Note that \(\mathcal{I}_{i}^{n} \cap \mathcal{I}_{j}^{n} = \emptyset\) whenever \(i \neq j\).
Letting \(\ell = \binom{n}{2}\) count the number of unique comparisons, we can now identify the fusion matrix by stacking the comparison matrices defined in (\ref{eq:comparison-matrix-1})-(\ref{eq:comparison-matrix-2}) (here given in terms of its transpose):
\begin{equation}
    \label{eq:fusion-matrix}
    \bD^{t}
    =
    \begin{bmatrix}
        \bD^{1,2}
        & \cdots
        & \bD^{i,j}
        & \cdots
        & \bD^{n-1,n}
    \end{bmatrix}^{t}.
\end{equation}
We are now in a position to attack the convex clustering problem with a distance penalty.
Letting \(S_{k}\) denote the set of \(k\)-block sparse vectors, we propose minimizing the penalized objective
\begin{equation}
    h_{\rho}(\bU;k)
    =
    \frac{1}{2}\|\bU - \bX\|_{F}^{2}
    +
    \frac{\rho}{2} \dist(\bD \vec(\bU), S_{k})^{2}
\end{equation}

\section*{\center Block Sparsity Projection}

Projection onto \(S_{k}\) is a simple operation.
We start with a motivating example in which blocks are of size 1.
Suppose we want a \(2\)-sparse representation of a vector \(\bx \in \Real^{5}\).
In this setting, we should keep that two largest entries of \(\bx\) and drop the rest in order to remain ``close'' to the original vector.
For example,
\begin{equation*}
    \begin{bmatrix}
        1 & 2 & 3 & 4 & 2
    \end{bmatrix}
    \overset{P_{S_{k}}}{\longrightarrow}
    \begin{bmatrix}
        0 & 0 & 3 & 4 & 0
    \end{bmatrix}
\end{equation*}
so that the distance between the two vectors is \(\sqrt{5}\).
Keeping the smallest components tends to increase the distance; see for example
\begin{equation*}
    \begin{bmatrix}
        1 & 2 & 3 & 4 & 2
    \end{bmatrix}
    {\longrightarrow}
    \begin{bmatrix}
        1 & 2 & 0 & 0 & 0
    \end{bmatrix}
\end{equation*}
which implies distance \(5\).
Extending this idea to structured sparsity requires us to have a notion of ``small blocks''.
A natural choice is to impose a norm condition.
For convex clustering, this means we should keep the top $k$ blocks based on \(\|\bu_{i} - \bu_{j}\|_{\dagger}\) and set the rest to zero.
This procedure effectively selects $k$ representatives for each cluster and assigns the remaining $n - k$ points to one of the \(k\) groups.

\begin{thebibliography}{1}
    \bibitem{chi2015}
    Chi, E. C., Lange, K. (2015). {Splitting Methods for Convex Clustering}. {Journal of Computational and Graphical Statistics}, 24(4), 994–1013. \url{https://doi.org/10.1080/10618600.2014.948181}
\end{thebibliography}
\end{document}