\documentclass{article}

\usepackage{amsbsy,amsthm,amsmath,amssymb}
%\usepackage{graphicx,natbib,booktabs}
\usepackage{graphicx}
\usepackage[margin=1in]{geometry}
\usepackage{subfig}
\usepackage{algorithm}
\usepackage{algpseudocode}
\usepackage{paralist}

\newtheorem{proposition}{Proposition}
\newtheorem{lemma}{Lemma}
\newtheorem{example}{Example}
\newtheorem{definition}{Definition}
\def\E{\mathop{\rm E\,\!}\nolimits}
\def\Var{\mathop{\rm Var}\nolimits}
\def\Cov{\mathop{\rm Cov}\nolimits}
\def\den{\mathop{\rm den}\nolimits}
\def\midd{\mathop{\,|\,}\nolimits}
\def\sgn{\mathop{\rm sgn}\nolimits}
\def\vec{\mathop{\rm vec}\nolimits}
\def\sinc{\mathop{\rm sinc}\nolimits}
\def\curl{\mathop{\rm curl}\nolimits}
\def\div{\mathop{\rm div}\nolimits}
\def\tr{\mathop{\rm tr}\nolimits}
\def\len{\mathop{\rm len}\nolimits}
\def\dist{\mathop{\rm dist}\nolimits}
\def\prox{\mathop{\rm prox}\nolimits}
\def\amp{\mathop{\:\:\,}\nolimits}
\def\Real{\mathop{\mathbb{R}}\nolimits}
\def\dom{\mathop{\bf dom}\nolimits}
\def\argmin{\mathop{\rm argmin}\nolimits}
\def\argmax{\mathop{\rm argmax}\nolimits}
\newcommand{\svskip}{\vspace{1.75mm}}
\newcommand{\ba}{\boldsymbol{a}}
\newcommand{\bb}{\boldsymbol{b}}
\newcommand{\bc}{\boldsymbol{c}}
\newcommand{\bd}{\boldsymbol{d}}
\newcommand{\be}{\boldsymbol{e}}
\newcommand{\bff}{\boldsymbol{f}}
\newcommand{\bg}{\boldsymbol{g}}
\newcommand{\bh}{\boldsymbol{h}}
\newcommand{\bi}{\boldsymbol{i}}
\newcommand{\bj}{\boldsymbol{j}}
\newcommand{\bk}{\boldsymbol{k}}
\newcommand{\bl}{\boldsymbol{l}}
\newcommand{\bm}{\boldsymbol{m}}
\newcommand{\bn}{\boldsymbol{n}}
\newcommand{\bo}{\boldsymbol{o}}
\newcommand{\bp}{\boldsymbol{p}}
\newcommand{\bq}{\boldsymbol{q}}
\newcommand{\br}{\boldsymbol{r}}
\newcommand{\bs}{\boldsymbol{s}}
\newcommand{\bt}{\boldsymbol{t}}
\newcommand{\bu}{\boldsymbol{u}}
\newcommand{\bv}{\boldsymbol{v}}
\newcommand{\bw}{\boldsymbol{w}}
\newcommand{\bx}{\boldsymbol{x}}
\newcommand{\by}{\boldsymbol{y}}
\newcommand{\bz}{\boldsymbol{z}}
\newcommand{\bA}{\boldsymbol{A}}
\newcommand{\bB}{\boldsymbol{B}}
\newcommand{\bC}{\boldsymbol{C}}
\newcommand{\bD}{\boldsymbol{D}}
\newcommand{\bE}{\boldsymbol{E}}
\newcommand{\bF}{\boldsymbol{F}}
\newcommand{\bG}{\boldsymbol{G}}
\newcommand{\bH}{\boldsymbol{H}}
\newcommand{\bI}{\boldsymbol{I}}
\newcommand{\bJ}{\boldsymbol{J}}
\newcommand{\bK}{\boldsymbol{K}}
\newcommand{\bL}{\boldsymbol{L}}
\newcommand{\bM}{\boldsymbol{M}}
\newcommand{\bN}{\boldsymbol{N}}
\newcommand{\bO}{\boldsymbol{O}}
\newcommand{\bP}{\boldsymbol{P}}
\newcommand{\bQ}{\boldsymbol{Q}}
\newcommand{\bR}{\boldsymbol{R}}
\newcommand{\bS}{\boldsymbol{S}}
\newcommand{\bT}{\boldsymbol{T}}
\newcommand{\bU}{\boldsymbol{U}}
\newcommand{\bV}{\boldsymbol{V}}
\newcommand{\bW}{\boldsymbol{W}}
\newcommand{\bX}{\boldsymbol{X}}
\newcommand{\bY}{\boldsymbol{Y}}
\newcommand{\bZ}{\boldsymbol{Z}}
\newcommand{\balpha}{\boldsymbol{\alpha}}
\newcommand{\bbeta}{\boldsymbol{\beta}}
\newcommand{\bgamma}{\boldsymbol{\gamma}}
\newcommand{\bdelta}{\boldsymbol{\delta}}
\newcommand{\bepsilon}{\boldsymbol{\epsilon}}
\newcommand{\blambda}{\boldsymbol{\lambda}}
\newcommand{\bmu}{\boldsymbol{\mu}}
\newcommand{\bnu}{\boldsymbol{\nu}}
\newcommand{\bphi}{\boldsymbol{\phi}}
\newcommand{\bpi}{\boldsymbol{\pi}}
\newcommand{\bsigma}{\boldsymbol{\sigma}}
\newcommand{\btheta}{\boldsymbol{\theta}}
\newcommand{\bomega}{\boldsymbol{\omega}}
\newcommand{\bxi}{\boldsymbol{\xi}}
\newcommand{\bGamma}{\boldsymbol{\rho}}
\newcommand{\bDelta}{\boldsymbol{\Delta}}
\newcommand{\bTheta}{\boldsymbol{\Theta}}
\newcommand{\bLambda}{\boldsymbol{\Lambda}}
\newcommand{\bXi}{\boldsymbol{\Xi}}
\newcommand{\bPi}{\boldsymbol{\Pi}}
\newcommand{\bOmega}{\boldsymbol{\Omega}}
\newcommand{\bUpsilon}{\boldsymbol{\Upsilon}}
\newcommand{\bPhi}{\boldsymbol{\Phi}}
\newcommand{\bPsi}{\boldsymbol{\Psi}}
\newcommand{\bSigma}{\boldsymbol{\Sigma}}
%% Matrix-matrix operations
\newcommand{\Kron}{\otimes} %Kronecker
%\newcommand{\Khat}{\odot} %Khatri-Rao
\newcommand{\Hada}{\ast} %Hadamard
%\newcommand{\Divide}{\varoslash}
%% Norms
\newcommand{\abs}[1]{\lvert{#1}\rvert}


\title{Proximal Distance Application: Metric Projection}
\author{Alfonso Landeros}
\date{\today}

\begin{document}
\maketitle

The nearest metric problem \cite{sra2005} can be attacked with the penalty method of optimization.
Given a \(n \times n\) dissimilarity matrix \(\bY = (y_{ij})\) with entries weighted by \(\bW = (w_{ij})\), our goal is to find a semi-metric \(\bX = (x_{ij})\) by solving a minimization problem of the form
\begin{equation*}
    h_{\rho}(\bX)
    =
    \frac{1}{2} \sum_{i>j} w_{ij}(x_{ij} - y_{ij})^{2}
    +
    \frac{\rho}{2} \sum_{i,j,k} c_{ijk}(\bX) + c_{jik}(\bX) + c_{kij}(\bX),
\end{equation*}
where the indices \((i,j,k)\) enumerate triangles.
The functions \(c_{\cdot \cdot \cdot}(\bX)\) denote one of three constraints associated with each triangle.
Symmetry allows one to store the matrices \(\bY\), \(\bW\), and \(\bX\) using entries from one of the strictly triangular regions.
It is therefore useful to define the ``\(\mathrm{trivec}\)'' operation which maps \(\bX\) to a vectorized version of itself, \(\bx\) (Figure \ref{fig:trivec-operation}).
The objective can then be expressed as
\begin{equation*}
  h_{\rho}(\bx)
  =
  \frac{1}{2} \lVert{\bW^{1/2}(\bx - \by)}\rVert^{2}_{2}
  +
  \frac{\rho}{2} \sum_{i,j,k} c_{ijk}(\bx) + c_{jik}(\bx) + c_{kij}(\bx).
\end{equation*}
\begin{figure}[!htbp]
    \centering
    \includegraphics[width=0.5\textwidth]{figures/trivec.pdf}
    \caption{
      Example of a symmetric matrix \(\bX\) and its vectorized counterpart \(\bx = \mathrm{trivec}(\bX)\).
    }
    \label{fig:trivec-operation}
  \end{figure}
Through an abuse of notation, the matrix \(\bW\) is now diagonal with entries \(w_{ij}\) and has the shape \(\binom{n}{2} \times \binom{n}{2}\).
It is also useful to introduce the constraint matrix \(\bT \in \Real^{p \times r}\) used to encode each triangle inequality where \(p = 3 \binom{n}{3}\) and \(r = \binom{n}{2}\).
Specifically, the row \(\bT_{ijk}\) is defined by its action on vectors;  \(\bT_{ijk} \bx = x_{i} - x_{j} - x_{k}\).
Our approach enforces non-negativity and metric constraints using distance functions
\begin{equation}
  \label{eq:objective}
  h_{\rho}(\bx)
  =
  \frac{1}{2} \lVert{\bW^{1/2} (\bx - \by)}\rVert^{2}_{2}
    +
  \frac{\rho}{2} \dist(\bT \bx, \mathbb{R}_{-}^{p})^{2}
    +
  \frac{\rho}{2} \dist(\bx, \mathbb{R}_{+}^{r})^{2},
\end{equation}
where \(\bx = \mathrm{trivec}(\bX)\) and \(\by = \mathrm{trivec}(\bY)\).
The first distance penalty represents the triangle inequality constraints \(\bT \bx \le {\bf 0}\) while the second term enforces non-negativity.
Fortunately, both distance functions have simple majorizations that yield a twice differentiable, strongly convex, and Lipschitz smooth surrogate
\begin{equation}
    \label{eq:surrogate}
    g_{\rho}(\bx \mid \bx_{n})
    =
    \frac{1}{2} \lVert{\bW^{1/2} (\bx - \by)}\rVert^{2}_{2}
    +
    \frac{\rho}{2} \lVert{\bT \bx - P(\bT \bx_{n}, \mathbb{R}_{-}^{p})}\rVert^{2}_{2}
    +
    \frac{\rho}{2} \lVert{\bx - P(\bx_{n}, \mathbb{R}_{+}^{r})}\rVert^{2}_{2}.
\end{equation}
The notation \(P(\cdot, S)\) denotes projection onto a set \(S\).
For convenience, we reduce the number of constriants by projecting onto the Cartesian product \(C = \Real_{+}^{r} \times \Real_{-}^{p}\) by defining the \textit{fusion matrix}
\(\bD = \begin{bmatrix}
    \bI & \bT^{\ast}
\end{bmatrix}^{\ast}\)
Thus, our surrogate~(\ref{eq:surrogate}) admits an equivalent representation
\begin{equation*}
    g_{\rho}(\bx \mid \bx_{n})
    =
    \frac{1}{2} \lVert{\bW^{1/2} (\bx - \by)}\rVert^{2}_{2}
    +
    \frac{\rho}{2} \lVert{\bD \bx - P(\bD \bx_{n}, C)}\rVert^{2}_{2},
\end{equation*}
with gradient
\begin{equation}
    \label{eq:surrogate-gradient}
    \bq_{n}
    =
    \bW (\bx - \by)
    +
    \rho \bD^{\ast} [\bD \bx - P(\bD \bx_{n}, C)].
\end{equation}

\section*{\center Structure of Constraints}

\begin{proposition}
    The matrix \(\bT\) has \(3 \binom{m}{3}\) rows and \(\binom{m}{2}\) columns.
\end{proposition}
\begin{proof}
    Interpret \(\bX\) as the adjacency matrix for a complete directed graph on \(m\) nodes without self-edges.
    When \(\bX\) is symmetric the number of free parameters is therefore \(\binom{m}{2}\).
    An oriented \(3\)-cycle is formed by fixing \(3\) nodes so there are \(\binom{m}{3}\) such cycles.
    Now fix the orientation of the \(3\)-cycles and note that each triangle encodes \(3\) metric constraints.
    The number of constraints is therefore \(3 \binom{m}{3}\).
\end{proof}

\begin{proposition}
    Each column of \(\bT\) has \(3 (m-2)\) nonzero entries.
\end{proposition}
\begin{proof}
    In view of the previous result, the entries \(T_{ij}\) encode whether edge \(j\) participates in constraint \(i\).
    We proceed by induction on the number of nodes \(m\).
    The base case \(m = 3\) involves one triangle and is trivial.
    Note that a triangle encodes \(3\) inequalities.

    Now consider a complete graph on \(m\) nodes and suppose the claim holds.
    Without loss of generality, consider the collection of \(3\)-cycles oriented clockwise and fix an edge \(j\).
    Adding a node to the graph yields \(2 m\) new edges, two for each of the existing \(m\) nodes.
    This action also creates one new triangle for each existing edge.
    Thus, edge \(j\) appears in \(3(m-2) + 3 = 3 [(m+1)-2]\) triangle inequality constraints based on the induction hypothesis.
\end{proof}
% TODO: Add a picture for "proof by picture".

\begin{proposition}
    Each column of \(\bT\) has \(m-2\) \(+1\)s and \(2(m-2)\) \(-1\)s.
\end{proposition}
\begin{proof}
Interpret the inequality \(x_{ij} \le x_{ik} + x_{kj}\) with \(i > k > j\) as the ordered triple \(x_{ij}, x_{ik}, x_{kj}\).
The statement is equivalent to counting
\begin{align*}
    a(N)
    &=
    \text{number of times \(x_{ij}\) appears in position 1, and}\\
    b(N)
    &=
    \text{number of times \(x_{ij}\) appears in position 2 or 3},
\end{align*}
where \(N\) denotes the number of constraints.
In view of the previous proposition, it is enough to prove \(a(N) = m-2\).
Note that \(a(3) = 1\), meaning that \(x_{ij}\) appears in position \(1\) exactly once within a given triangle.
Given that an edge \((i,j)\) appears in \(3 (m-2)\) constraints, divide this quantity by the number of constraints per triangle to arrive at the stated result.
\end{proof}

\begin{proposition}
    The matrix \(\bT\) has full column rank.
\end{proposition}
\begin{proof}
    It is enough to show that \(\bA = \bT^{\ast}\bT\) is full rank.
    The first two propositions imply
    \begin{equation*}
        a_{ii}
        =
        \langle{\bT_{i}, \bT_{i}}\rangle
        =
        \sum (\pm 1)^{2}
        =
        3 (m-2).
    \end{equation*}
    To compute the off-diagonal entries, fix a triangle and note that two edges \(i\) and \(j\) appear in all three of its constraints of the form \(x_{i} \le x_{j} + x_{k}\).
    There are three possibilities for a given constraint \(c\):
    \begin{equation*}
        T_{c,i} T_{c,j} \amp = \amp
        \begin{cases}
            -1, &\text{if \(i\) LHS, \(j\) RHS or vice-versa} \\
            \amp 1,  &\text{if \(i\) and \(j\) both appear on RHS} \\
            \amp 0,  &\text{if one of \(i\) or \(j\) is missing}.
        \end{cases}
    \end{equation*}
    It follows that
    \begin{equation*}
        a_{ij}
        =
        \langle{\bT_{i}, \bT_{j}}\rangle
        =
        \begin{cases}
            -1, &\text{if}\ldots \\
            \amp 0, &\text{otherwise}.
        \end{cases}
    \end{equation*}
    By Proposition 2, an edge \(i\) appears in \(3 (m-2)\) constraints.
    Imposing the condition that edge \(j\) also appears reduces this number by \(m-2\), the number of remaining nodes that can contribute edges in our accounting.
    The calculation
    \begin{equation*}
        \sum_{j \neq i} |a_{ij}| = 2 (m-2) < 3 (m-2) = |a_{ii}|
    \end{equation*}
    establishes that \(\bA\) is strictly diagonally dominant and hence full rank.
\end{proof}

\begin{proposition}
    The matrix \(\bT^{\ast}\bT\) has at most \(3\) distinct eigenvalues of the form \(m-2\), \(2m-2\), and \(3m-4\) with multiplicities \(1\), \(m-1\), and \(\frac{1}{2} m (m-3)\), respectively.
\end{proposition}

\section*{\center Required Operators}

Let \(\bz = \bT^{\ast} \bT \bx - \bT^{\ast} (\bT \bx)_{+}\) with \(\bz = \vec(\bZ)\).
One can compute the required operation with a matrix-free algorithm in a single loop in \(\mathcal{O}(n^{3})\) operations.
The recipe accumulates
\begin{equation*}
    \begin{split}
        z_{ij}
        &=
        z_{ij}
        + 3 x_{ij} - x_{ki} - x_{kj}
        - (x_{ij})_{-} + (x_{ki})_{-} + (x_{kj})_{-}
        \qquad j = 1,\ldots,n-2 \\
        z_{ki}
        &=
        z_{ki}
        + 3 x_{ki} - x_{ij} - x_{kj}
        - (x_{ij})_{-} + (x_{ki})_{-} + (x_{kj})_{-}
        \qquad i = j+1,\ldots,n-1 \\
        z_{kj}
        &=
        z_{kj}
        + 3 x_{kj} - x_{ij} - x_{ki}
        - (x_{ij})_{-} + (x_{ki})_{-} + (x_{kj})_{-}
        \qquad k = i+1,\ldots,n
    \end{split}
\end{equation*}
by visiting each triangle \((i,j,k)\) exactly once.
Note that the algorithm operates on two columns at a time.
Computing \(\bw = \bT \bx\) has a similar computational complexity:
\begin{equation*}
    \begin{split}
        y_{ijk} &= x_{ij} - x_{ki} - x_{kj} \\
        y_{jik} &= x_{ki} - x_{ij} - x_{kj} \\
        y_{kij} &= x_{kj} - x_{ij} - x_{ki}. \\
    \end{split}
\end{equation*}
Numerical experiments suggest that these two algorithms are superior to explicit matrix-vector with sparse matrix data structures despite \(>99\%\) sparsity for \(n \gg 1\).
This is likely due to the large bandwidth of both matrices \(\bT\) and \(\bT^{\ast} \bT\).

\section*{\center Algorithm 1: Proximal Distance}

The stationary condition \(\nabla g_{\rho}(\bx \mid \bx_{n}) = {\bf 0}\) produces the iteration scheme
\begin{equation*}
  \begin{split}
    \bx_{n+1}
    &=
    \left(
      \frac{1}{\rho} \bW + \bI + \bT^{\ast} \bT
    \right)^{-1}
    \left[
      \frac{1}{\rho} \bW \by
      +
      \bT^{\ast} (\bT \bx_{n})_{-}
      +
      (\bx_{n})_{+}
    \right] \\
    &=
    \rho [\nabla^{2} g_{\rho}(\bx \mid \bx_{n})]^{-1}
    \left[
      \frac{1}{\rho} \bW \by
      +
      \bT^{\ast} (\bT \bx_{n})_{-}
      +
      (\bx_{n})_{+}
    \right].
  \end{split}
\end{equation*}
The Gram matrix \(\bT^{\ast} \bT\) preserves sparsity because rows of \(\bT\) are sparse except when \(m \le 4\).
Moreover, one can avoid repeated evaluation of the requisite matrix inverse by computing its spectral decomposition.
Woodbury's identity can further simplify matters:
\begin{equation}
  \label{eq:matrix-inverse}
  \begin{split}
    \left(
      \frac{1}{\rho} \bW + \bI + \bT^{\ast} \bT
    \right)^{-1}
    &= \cdots
    % &=
    % \frac{1}{\rho} \bW
    % -
    % \left(\frac{1}{\rho}\right)^{2} \bV \left(
    %   \bLambda^{-1} + \frac{1}{\rho} \bW
    % \right)^{-1} \bV^{-1} \\
    % &=
    % \frac{1}{\rho} \bW
    % -
    % \left(\frac{1}{\rho}\right)^{2} \bV \bZ \bV^{-1},
  \end{split}
\end{equation}
% and identify the entries \(z_{ii} = \lambda_{ii} (\frac{1}{\rho} \lambda_{ii} + w_{ii})^{-1}\) of the diagonal matrix \(\bZ\).
Unfortunately, the number of parameters is \(\binom{n}{2}\) so the cost of the decomposition is  \(\mathcal{O}(n^{6})\).
Savings from caching the decomposition become marginal as the number of nodes increases.
One can show that the positive definite matrix \(\bT^{\ast} \bT\) possesses three distinct eigenvalues.
However, the eigenvectors spanning each eigenspace are difficult to describe with exact formulas.
In lieu of the explicit eigenstructure, one can take advantage of sparsity by using the method of conjugate gradients to solve the required linear system.

Observe that \(h_{\rho}(\bx)\) is strongly convex for fixed \(\rho\) and possesses a global minimum in the limit \(\rho \to \infty\) provided the penalty terms vanish.
Unfortunately, the Hessian of the surrogate, \(g_{\infty}(\bx \mid \bx_{\infty})\), is not bounded so it is not obvious that the MM iterates converge to a limit point.
The equality
\begin{equation*}
  \bx_{\infty}
  =
  \left(
      \bI + \bT^{\ast} \bT
    \right)^{-1}
    \left[
      \bT^{\ast} (\bT \bx_{\infty})_{-}
      +
      (\bx_{\infty})_{+}
    \right],
\end{equation*}
holds precisely when the putative limit point \(\bx_{\infty}\) respects both constraints.
Hence, the MM iterates \(\{\bx_{n}\}_{n \ge 0}\) converge to a no-progress point.

\section*{\center Algorithm 2: Proximal Distance + Steepest Descent}

Linearity of the gradient~(\ref{eq:surrogate-gradient}) suggests an explicit formula for the optimal step size \(\gamma_{n}\) along \(\bq_{n} = \nabla g_{\rho}(\bx \mid \bx_{n})\).
This follows easily from the chain rule
\begin{equation*}
  \begin{split}
    \frac{\partial}{\partial \gamma} g_{\rho}(\bx - \gamma \bq_{n} \mid \bx_{n})
    &=
    \langle{\nabla g_{\rho}(\bx - \gamma \bq \mid \bx_{n}), \bq_{n}}\rangle \\
    &=
    \langle{
      \bq_{n}
        - \gamma \bW \bq_{n}
        - \gamma \rho \bD^{\ast} \bD \bq_{n},
     \bq_{n}}\rangle \\
    &=
    \lVert{\bq_{n}}\rVert^{2}
    - \gamma (\lVert{\bW^{1/2} \bq_{n}}\rVert^{2} + \rho \lVert{\bT \bq_{n}}\rVert^{2}) \\
    \implies \gamma_{n}
    &=
    \frac{\lVert{\bq_{n}}\rVert^{2}}{
        \lVert{\bW^{1/2} \bq_{n}}\rVert^{2}
        + \rho \lVert{\bq_{n}}\rVert^{2}
        + \rho \lVert{\bT \bq_{n}}\rVert^{2}
    }
  \end{split}
\end{equation*}
Explicitly, the iterative algorithm is given by
\begin{equation*}
    \bx_{n+1}
    =
    \bx_{n}
    - \gamma \left[
    \bW (\bx - \by) + \rho \{
        \bx - (\bx_{n})_{+}
        + \bT^{\ast} \bT \bx - \bT^{\ast} (\bT \bx_{n})_{-}
        \}
    \right].
\end{equation*}
%%%%% BIBLIOGRAPHY %%%%%
\begin{thebibliography}{99}
    \bibitem{sra2005}
    Sra S, Tropp J, Dhillon IS (2005) {Triangle fixing algorithms for the metric nearness problem}. {\it Advances in Neural Information Processing Systems} 361-368
    
    \bibitem{langeMM2016}
    Lange, K (2016) {\it MM Optimization Algorithms}. SIAM
\end{thebibliography}
\end{document}