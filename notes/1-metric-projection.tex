\documentclass{article}

\usepackage{amsbsy,amsthm,amsmath,amssymb}
%\usepackage{graphicx,natbib,booktabs}
\usepackage{graphicx}
\usepackage[margin=1in]{geometry}
\usepackage{subfig}
\usepackage{algorithm}
\usepackage{algpseudocode}
\usepackage{paralist}

\newtheorem{proposition}{Proposition}
\newtheorem{lemma}{Lemma}
\newtheorem{example}{Example}
\newtheorem{definition}{Definition}
\def\E{\mathop{\rm E\,\!}\nolimits}
\def\Var{\mathop{\rm Var}\nolimits}
\def\Cov{\mathop{\rm Cov}\nolimits}
\def\den{\mathop{\rm den}\nolimits}
\def\midd{\mathop{\,|\,}\nolimits}
\def\sgn{\mathop{\rm sgn}\nolimits}
\def\vec{\mathop{\rm vec}\nolimits}
\def\sinc{\mathop{\rm sinc}\nolimits}
\def\curl{\mathop{\rm curl}\nolimits}
\def\div{\mathop{\rm div}\nolimits}
\def\tr{\mathop{\rm tr}\nolimits}
\def\len{\mathop{\rm len}\nolimits}
\def\dist{\mathop{\rm dist}\nolimits}
\def\prox{\mathop{\rm prox}\nolimits}
\def\amp{\mathop{\:\:\,}\nolimits}
\def\Real{\mathop{\mathbb{R}}\nolimits}
\def\dom{\mathop{\bf dom}\nolimits}
\def\argmin{\mathop{\rm argmin}\nolimits}
\def\argmax{\mathop{\rm argmax}\nolimits}
\newcommand{\svskip}{\vspace{1.75mm}}
\newcommand{\ba}{\boldsymbol{a}}
\newcommand{\bb}{\boldsymbol{b}}
\newcommand{\bc}{\boldsymbol{c}}
\newcommand{\bd}{\boldsymbol{d}}
\newcommand{\be}{\boldsymbol{e}}
\newcommand{\bff}{\boldsymbol{f}}
\newcommand{\bg}{\boldsymbol{g}}
\newcommand{\bh}{\boldsymbol{h}}
\newcommand{\bi}{\boldsymbol{i}}
\newcommand{\bj}{\boldsymbol{j}}
\newcommand{\bk}{\boldsymbol{k}}
\newcommand{\bl}{\boldsymbol{l}}
\newcommand{\bm}{\boldsymbol{m}}
\newcommand{\bn}{\boldsymbol{n}}
\newcommand{\bo}{\boldsymbol{o}}
\newcommand{\bp}{\boldsymbol{p}}
\newcommand{\bq}{\boldsymbol{q}}
\newcommand{\br}{\boldsymbol{r}}
\newcommand{\bs}{\boldsymbol{s}}
\newcommand{\bt}{\boldsymbol{t}}
\newcommand{\bu}{\boldsymbol{u}}
\newcommand{\bv}{\boldsymbol{v}}
\newcommand{\bw}{\boldsymbol{w}}
\newcommand{\bx}{\boldsymbol{x}}
\newcommand{\by}{\boldsymbol{y}}
\newcommand{\bz}{\boldsymbol{z}}
\newcommand{\bA}{\boldsymbol{A}}
\newcommand{\bB}{\boldsymbol{B}}
\newcommand{\bC}{\boldsymbol{C}}
\newcommand{\bD}{\boldsymbol{D}}
\newcommand{\bE}{\boldsymbol{E}}
\newcommand{\bF}{\boldsymbol{F}}
\newcommand{\bG}{\boldsymbol{G}}
\newcommand{\bH}{\boldsymbol{H}}
\newcommand{\bI}{\boldsymbol{I}}
\newcommand{\bJ}{\boldsymbol{J}}
\newcommand{\bK}{\boldsymbol{K}}
\newcommand{\bL}{\boldsymbol{L}}
\newcommand{\bM}{\boldsymbol{M}}
\newcommand{\bN}{\boldsymbol{N}}
\newcommand{\bO}{\boldsymbol{O}}
\newcommand{\bP}{\boldsymbol{P}}
\newcommand{\bQ}{\boldsymbol{Q}}
\newcommand{\bR}{\boldsymbol{R}}
\newcommand{\bS}{\boldsymbol{S}}
\newcommand{\bT}{\boldsymbol{T}}
\newcommand{\bU}{\boldsymbol{U}}
\newcommand{\bV}{\boldsymbol{V}}
\newcommand{\bW}{\boldsymbol{W}}
\newcommand{\bX}{\boldsymbol{X}}
\newcommand{\bY}{\boldsymbol{Y}}
\newcommand{\bZ}{\boldsymbol{Z}}
\newcommand{\balpha}{\boldsymbol{\alpha}}
\newcommand{\bbeta}{\boldsymbol{\beta}}
\newcommand{\bgamma}{\boldsymbol{\gamma}}
\newcommand{\bdelta}{\boldsymbol{\delta}}
\newcommand{\bepsilon}{\boldsymbol{\epsilon}}
\newcommand{\blambda}{\boldsymbol{\lambda}}
\newcommand{\bmu}{\boldsymbol{\mu}}
\newcommand{\bnu}{\boldsymbol{\nu}}
\newcommand{\bphi}{\boldsymbol{\phi}}
\newcommand{\bpi}{\boldsymbol{\pi}}
\newcommand{\bsigma}{\boldsymbol{\sigma}}
\newcommand{\btheta}{\boldsymbol{\theta}}
\newcommand{\bomega}{\boldsymbol{\omega}}
\newcommand{\bxi}{\boldsymbol{\xi}}
\newcommand{\bGamma}{\boldsymbol{\rho}}
\newcommand{\bDelta}{\boldsymbol{\Delta}}
\newcommand{\bTheta}{\boldsymbol{\Theta}}
\newcommand{\bLambda}{\boldsymbol{\Lambda}}
\newcommand{\bXi}{\boldsymbol{\Xi}}
\newcommand{\bPi}{\boldsymbol{\Pi}}
\newcommand{\bOmega}{\boldsymbol{\Omega}}
\newcommand{\bUpsilon}{\boldsymbol{\Upsilon}}
\newcommand{\bPhi}{\boldsymbol{\Phi}}
\newcommand{\bPsi}{\boldsymbol{\Psi}}
\newcommand{\bSigma}{\boldsymbol{\Sigma}}
%% Matrix-matrix operations
\newcommand{\Kron}{\otimes} %Kronecker
%\newcommand{\Khat}{\odot} %Khatri-Rao
\newcommand{\Hada}{\ast} %Hadamard
%\newcommand{\Divide}{\varoslash}
%% Norms
\newcommand{\abs}[1]{\lvert{#1}\rvert}

\usepackage{booktabs}

\title{Example 1: Metric Projection}
\author{}

\begin{document}
\maketitle

\section*{\center Model}

Given a \(n \times n\) dissimilarity matrix $\bC = (c_{ij})$ with non-negative weights $w_{ij}$, our goal is to find a semi-metric \(\bX = (x_{ij})\) by solving the constrainted problem
\begin{equation*}
    f(\bX)
    =
    \frac{1}{2} \sum_{i>j} w_{ij}(x_{ij} - c_{ij})^{2}
    \qquad
    \text{subject to}
    \quad
    x_{ij} - x_{ik} \le x_{kj},
    \quad
    x_{ij} \ge 0.
\end{equation*}
The $\mathrm{trivec}$ operation admits a minimal representation of \(\bX\), \(\bx = \mathrm{trivec}(\bX)\) (Figure \ref{fig:trivec-operation}).
In the proximal distance framework, we work with the objective
\begin{equation*}
  h_{\rho}(\bx)
  =
  \frac{1}{2} \|\bW^{1/2}(\bx - \bc)\|^{2}_{2}
  + \frac{\rho}{2} \dist(\bT \bx, \Real_{+}^{m_{1}})
  + \frac{\rho}{2} \dist(\bx, \Real_{+}^{m_{2}})
\end{equation*}
\begin{figure}[!htbp]
    \centering
    \includegraphics[width=0.5\textwidth]{figures/trivec.pdf}
    \caption{
      Example of a symmetric matrix \(\bX\) and its minimal representation \(\bx = \mathrm{trivec}(\bX)\).
    }
    \label{fig:trivec-operation}
  \end{figure}
Here $\bT$ encodes triangle inequalities and the $m_{i}$ count the number of contraints of each type.
The usual distance majorization furnishes a surrogate
\begin{align*}
    g_{\rho}(\bx \mid \bx_{n})
    &=
    \frac{1}{2} \|\bW^{1/2} (\bx - \bc)\|^{2}_{2}
    +
    \frac{\rho}{2} \|\bT \bx - \mathcal{P}(\bT \bx_{n}, \mathbb{R}_{+}^{m_{1}})\|^{2}_{2}
    +
    \frac{\rho}{2} \lVert{\bx - \mathcal{P}(\bx_{n}, \mathbb{R}_{+}^{m_{2}})}\rVert^{2}_{2} \\
    &=
    \frac{1}{2} \|\bW^{1/2} (\bx - \bc)\|^{2}_{2}
    +
    \frac{\rho}{2} \|\bD \bx - \mathcal{P}(\bD \bx_{n})\|_{2}^{2}
\end{align*}
The notation \(\mathcal{P}(\cdot, S)\) denotes projection onto a set \(S\).
The fusion matrix $\bD = [\bT; \bI]$ stacks the two operators; the joint projection operates in a block-wise fashion.

\section*{\center Algorithm Maps}

\subsection*{MM}

Rewrite the surrogate explicitly a least squares problem minimizing $\|\bA \bx - \bb_{n}\|^{2}_{2}$:
\begin{equation*}
  \bx_{n+1} = \underset{\bx}{\argmin} \frac{1}{2} \left\|
    \begin{bmatrix}
      \bW^{1/2} \\
      \sqrt{\rho} \bD
    \end{bmatrix} \bx
    -
    \begin{bmatrix}
      \bc \\
      \sqrt{\rho} \mathcal{P}(\bD \bx_{n})
    \end{bmatrix}
  \right\|_{2}^{2}
\end{equation*}
Updating the RHS $\bb_{n}$ in the linear system reduces to evaluating the projection and copy operations.
It is worth noting that triangle fixing algorithms that solve the metric nearness problem operate in the same fashion, except they work one triangle a time.
That is, each iteration solves $\binom{n}{3}$ least squares problem compared to 1 in this formulation.
A conjugate gradient type algorithm solves the normal equations directly using $\bA^{t}\bA$, whereas LSQR type methods use only $\bA$ and $\bA^{t}$.

\subsection*{Steepest Descent}

The updates $\bx_{n+1} = \bx_{n} - \gamma_{n} \nabla h_{\rho}(\bx_{n})$ admit an exact solution for the line search parameter $\gamma_{n}$.
Recall the generic formula from the main text ($\gamma_{n} \equiv t_{n}$ and $\bq_{n} \equiv \bv_{n}$ to match notation):
\begin{equation*}
  \gamma_{n} = \frac{\|\bq_{n}\|^{2}}{\bq_{n}^{t}\bA\bq_{n} + \rho \|\bD \bq_{n}\|^{2}}
\end{equation*}
Identifying $\bq_{n}$ with $\nabla h_{\rho}(\bx_{n})$ we have
\begin{align*}
  \nabla h_{\rho}(\bx_{n})
  &= \bW(\bx_{n} - \bc) + \rho \bD^{t} [\bD \bx_{n} - \mathcal{P}(\bD \bx_{n})] \\
  &= \bW(\bx_{n} - \bc) + \rho (\bI + \bT^{t}\bT) \bx_{n} - \rho[\bT^{t} \mathcal{P}(\bT \bx_{n}, \Real_{+}^{m_{1}}) + \mathcal{P}(\bx_{n}, \Real_{+}^{m_{1}})] \\
  \gamma_{n}
  &=
  \frac{\|\bq_{n}\|^{2}}{\|\bW^{1/2} \bq\|^{2} + \rho \|\bD \bq\|^{2}}.
\end{align*}

\subsection*{ADMM}

Taking $\by$ as the dual variable and $\blambda$ as scaled multipliers, the updates for each ADMM block are

\begin{align*}
  \bx_{n+1}
  &= \underset{\bx}{\argmin} \left\|
    \begin{bmatrix}
      \bW^{1/2} \\
      \sqrt{\mu} \bD
    \end{bmatrix} \bx
    -
    \begin{bmatrix}
      \bc \\
      \sqrt{\mu} (\by_{n} - \blambda_{n}))
    \end{bmatrix}
  \right\|_{2}^{2} \\
  \by_{n+1}
  &= \frac{\alpha}{1+\alpha} \mathcal{P}(\bz_{n}) + \frac{1}{1+\alpha} \bz_{n};
  \qquad \bz_{n} = \bD \bx_{n+1} + \blambda_{n},~\alpha = \rho / \mu
\end{align*}
Multipliers follow the standard update.

\section*{\center Properties of the Triangle Inequality Matrix}

These results were documented by Sra et al. in a technical report.
Here the symbol $m$ counts the number of nodes/measurements in the problem.

\begin{proposition}
    The matrix \(\bT\) has \(3 \binom{m}{3}\) rows and \(\binom{m}{2}\) columns.
\end{proposition}
\begin{proof}
    Interpret \(\bX\) as the adjacency matrix for a complete directed graph on \(m\) nodes without self-edges.
    When \(\bX\) is symmetric the number of free parameters is therefore \(\binom{m}{2}\).
    An oriented \(3\)-cycle is formed by fixing \(3\) nodes so there are \(\binom{m}{3}\) such cycles.
    Now fix the orientation of the \(3\)-cycles and note that each triangle encodes \(3\) metric constraints.
    The number of constraints is therefore \(3 \binom{m}{3}\).
\end{proof}

\begin{proposition}
    Each column of \(\bT\) has \(3 (m-2)\) nonzero entries.
\end{proposition}
\begin{proof}
    In view of the previous result, the entries \(T_{ij}\) encode whether edge \(j\) participates in constraint \(i\).
    We proceed by induction on the number of nodes \(m\).
    The base case \(m = 3\) involves one triangle and is trivial.
    Note that a triangle encodes \(3\) inequalities.

    Now consider a complete graph on \(m\) nodes and suppose the claim holds.
    Without loss of generality, consider the collection of \(3\)-cycles oriented clockwise and fix an edge \(j\).
    Adding a node to the graph yields \(2 m\) new edges, two for each of the existing \(m\) nodes.
    This action also creates one new triangle for each existing edge.
    Thus, edge \(j\) appears in \(3(m-2) + 3 = 3 [(m+1)-2]\) triangle inequality constraints based on the induction hypothesis.
\end{proof}
% TODO: Add a picture for "proof by picture".

\begin{proposition}
    Each column of \(\bT\) has \(m-2\) \(+1\)s and \(2(m-2)\) \(-1\)s.
\end{proposition}
\begin{proof}
Interpret the inequality \(x_{ij} \le x_{ik} + x_{kj}\) with \(i > k > j\) as the ordered triple \(x_{ij}, x_{ik}, x_{kj}\).
The statement is equivalent to counting
\begin{align*}
    a(N)
    &=
    \text{number of times \(x_{ij}\) appears in position 1, and}\\
    b(N)
    &=
    \text{number of times \(x_{ij}\) appears in position 2 or 3},
\end{align*}
where \(N\) denotes the number of constraints.
In view of the previous proposition, it is enough to prove \(a(N) = m-2\).
Note that \(a(3) = 1\), meaning that \(x_{ij}\) appears in position \(1\) exactly once within a given triangle.
Given that an edge \((i,j)\) appears in \(3 (m-2)\) constraints, divide this quantity by the number of constraints per triangle to arrive at the stated result.
\end{proof}

\begin{proposition}
    The matrix \(\bT\) has full column rank.
\end{proposition}
\begin{proof}
    It is enough to show that \(\bA = \bT^{\ast}\bT\) is full rank.
    The first two propositions imply
    \begin{equation*}
        a_{ii}
        =
        \langle{\bT_{i}, \bT_{i}}\rangle
        =
        \sum (\pm 1)^{2}
        =
        3 (m-2).
    \end{equation*}
    To compute the off-diagonal entries, fix a triangle and note that two edges \(i\) and \(j\) appear in all three of its constraints of the form \(x_{i} \le x_{j} + x_{k}\).
    There are three possibilities for a given constraint \(c\):
    \begin{equation*}
        T_{c,i} T_{c,j} \amp = \amp
        \begin{cases}
            -1, &\text{if \(i\) LHS, \(j\) RHS or vice-versa} \\
            \amp 1,  &\text{if \(i\) and \(j\) both appear on RHS} \\
            \amp 0,  &\text{if one of \(i\) or \(j\) is missing}.
        \end{cases}
    \end{equation*}
    It follows that
    \begin{equation*}
        a_{ij}
        =
        \langle{\bT_{i}, \bT_{j}}\rangle
        =
        \begin{cases}
            -1, &\text{if}\ldots \\
            \amp 0, &\text{otherwise}.
        \end{cases}
    \end{equation*}
    By Proposition 2, an edge \(i\) appears in \(3 (m-2)\) constraints.
    Imposing the condition that edge \(j\) also appears reduces this number by \(m-2\), the number of remaining nodes that can contribute edges in our accounting.
    The calculation
    \begin{equation*}
        \sum_{j \neq i} |a_{ij}| = 2 (m-2) < 3 (m-2) = |a_{ii}|
    \end{equation*}
    establishes that \(\bA\) is strictly diagonally dominant and hence full rank.
\end{proof}

\begin{proposition}
    The matrix \(\bT^{\ast}\bT\) has at most \(3\) distinct eigenvalues of the form \(m-2\), \(2m-2\), and \(3m-4\) with multiplicities \(1\), \(m-1\), and \(\frac{1}{2} m (m-3)\), respectively.
\end{proposition}

%%%%% BIBLIOGRAPHY %%%%%
\begin{thebibliography}{99}
    \bibitem{sra2005}
    Sra S, Tropp J, Dhillon IS (2005) {Triangle fixing algorithms for the metric nearness problem}. {\it Advances in Neural Information Processing Systems} 361-368
\end{thebibliography}
\end{document}
